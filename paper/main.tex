% !TeX spellcheck = en_UK
\documentclass[10pt, a4paper, titlepage, twocolumn]{article}

\usepackage[utf8]{inputenc}
\usepackage[english]{babel}
\usepackage{amsmath}
\usepackage{amsfonts}
\usepackage{amssymb}
\usepackage{graphicx}
\usepackage{url}
\usepackage{hyperref}
\usepackage{ragged2e}
\usepackage{tocbibind}
\usepackage{float}
\usepackage{placeins}
\usepackage{fancyhdr}
\usepackage{datetime}
\usepackage[dvipsnames,table]{xcolor}
\usepackage[backend=bibtex, style=numeric, sorting=none, maxbibnames=6, style=ieee]{biblatex}
\usepackage{geometry}
\usepackage[pages=some]{background}
\usepackage{setspace}
\usepackage{anysize}
\usepackage{smartdiagram}
\usepackage{tabulary}
\usepackage{booktabs}
\usepackage{multirow}
\usepackage{wrapfig}
\usepackage{csquotes}
\usepackage{longtable}
\usepackage{tocloft}
\usepackage{appendix}
\usepackage{multicol}

\author{}
\title{}
\date{Córdoba, \today}

\newdateformat{monthyeardate}{\monthname[\THEMONTH] de \THEYEAR}

%marginsize{left}{right}{top}{bottom}
\marginsize{2cm}{2cm}{2.5cm}{3.5cm}

\addbibresource{main.bib}

\setlength{\columnsep}{0.7cm}

\begin{document}
	\fontfamily{lmr}\selectfont
		
	\section{Introduction}
	\label{sect:introduction}
	Deep learning \cite{lecun2015deep} combines multiple Machine Learning techniques and allows computational models that are composed of numerous processing layers to learn representations of data with various levels of abstraction. These methods have dramatically improved the state-of-the-art in many domains, such as image classification or speech recognition. Deep learning discovers complex structures in large datasets by using the backpropagation algorithm to indicate how a machine should change its internal parameters that are used to compute the representation in each layer from the features in the previous layer.
	
	The backpropagation algorithm is a stochastic gradient descent algorithm that minimises a predefined loss function. The layer's parameters are updated after backpropagating the loss function gradients through the network. Deep convolutional nets are suited for images, video, speech and audio processing and they have been used extensively in the last years for automatic classification tasks. For the case of images, convolutional layers extract the main features from the pixels of the images and, after that, the fully connected layer classify every sample based on its extracted features. At the output of the convolutional net, a softmax function provides the probabilities of the set of classes predefined in the model for classification tasks. These outputs are compared against the correct values.
	
	Multiple metrics exist in the literature of machine learning and statistics \cite{cruz2014metrics}\cite{mehdiyev2016evaluating}. Kappa index is a well-known statistic coefficient defined by Cohen \cite{cohen1960coefficient} to measure inter-rater agreement on classifying elements into a set of categories. Later, Weighted Kappa (WK) is a modified version of the Kappa statistic calculated to allow assigning different weights to different levels of aggregation between two variables. Weighted Kappa loss was described in previous work \cite{de2018weighted} and is a cost function that is based on the WK metric. These functions are indicated for problems where different inter-classes weights are assigned, and that work proved that using these functions improves the model performance and reduce the overfitting risk. This kind of problems is associated with ordinal data processing. Those weights are predefined and depend on the type of the chosen WK. In a linear penalisation, the weight is proportional to the distance between the predicted and the real class. In the Quadratic Weighted Kappa (QWK) the penalisation is proportional to the square of the distance. In this work, we will combine the QWK metric and the QWK loss.
	
	Softmax function is widely used for the output layer in neural networks for classification tasks. It is a simple and efficient function for multi-class problems but not the best when working with ordinal data. In this paper, the Proportional Odds Model (POM) will be used instead of the softmax function. Different link functions will be explored to compare their performance. Also, the influence of other parameters like the learning rate of the optimization algorithm and the batch size will be studied. We will contrast the results obtained with statistical analysis to provide more robust results.
	
	The experiments will be run using two different ordinal datasets: Diabetic Retinopathy and Adience.
	
	This paper is organised as follows: in Section \ref{sect:relatedwork}, we take a look at previous works related to this paper, in Section \ref{sect:experiments} we describe the model, the experiments and the datasets used, in Section \ref{sect:results} we present the results obtained and the statistical analysis and finally in Section \ref{sect:conclusions} we expose the conclusions of this study.
	
	\section{Related work}
	\label{sect:relatedwork}
	J. de la Torre et al. proposed the use of QWK loss function for the optimization algorithm. They compared this cost function against the traditional log-loss function across three different databases, including the Diabetic Retinopathy database as the most complex one. They proved that their function could improve the results as it reduces the overfitting and the required training time. Also, they checked the importance of hyper-parameter tuning.

	
	\section{Experiments}
	\label{sect:experiments}
	\subsection{Data}
	We make use of two ordinal datasets appropriate for deep neural networks:
	
	\begin{itemize}
		\item \textit{Diabetic Retinopathy (DR)}\footnote{https://www.kaggle.com/c/diabetic-retinopathy-detection/data}. DR is a dataset consisting of extremely high-resolution fundus image data. The training set consists of $17563$ pairs of images (where a  pair consists of a left and right eye image corresponding to a patient). In this dataset, we try and predict from five levels of diabetic retinopathy: no DR ($25810$ images), mild DR ($2443$ images), moderate DR ($5292$ images), severe DR ($873$ images), or proliferative DR ($708$ images). The images are taken in variable conditions: by different cameras, illumination conditions and resolutions. These images come from the EyePACS dataset that was used in a Diabetic Retinopathy Detection competition that was hosted on the Kaggle platform. Also, this dataset was used in later works \cite{de2018weighted}\cite{nebot2016diabetic} and ordinal techniques (such as an ordinal cost function) were applied in order to achieve better performance. A validation set is set aside, consisting of $10\%$ of the patients in the training set. The images are resized to 128 by 128 pixels. Data augmentation techniques are applied to achieve a higher number of samples.
		
		\item \textit{Adience}\footnote{http://www.openu.ac.il/home/hassner/Adience/data.html}. This dataset consists of $26580$ faces belonging to $2284$ subjects. We use the form of the dataset where faces have been pre-cropped and aligned. The dataset was preprocessed, using the methods described in a previous work \cite{beckham2017unimodal}, so that the images are 256 pixels in width and height, and pixels values follow a $(0;1)$ normal distribution. The original dataset is split into five cross-validation folds. The training set consists of merging the first four folds which comprise a total of $15554$ images. From this, $10\%$ of the images are held out as part of a validation set. The last fold is used as test set.
	\end{itemize}
	
	\subsection{The model}
	A convolutional neural network (CNN) has been used for both datasets. The architecture of this CNN is presented in the Table \ref{table:CNNArchitecture}.
	
	\begin{table}[ht]
		\centering
		\begin{tabular}{|l|c|}
			\hline
			\textbf{Layer} & \textbf{Output shape}\\
			\hline
			Conv\_32\_3x3 & 254x254x32\\\hline
			Conv\_32\_3x3 & 252x252x32\\\hline
			MaxPool\_2x2 & 126x126x32\\\hline
			
			Conv\_64\_3x3 & 124x124x64\\\hline
			Conv\_64\_3x3 & 122x122x64\\\hline
			MaxPool\_2x2 & 61x61x64\\\hline
			
			Conv\_128\_3x3 & 59x59x128\\\hline
			Conv\_128\_3x3 & 57x57x128\\\hline
			MaxPool\_2x2 & 28x28x128\\\hline
			
			Conv\_128\_3x3 & 26x26x128\\\hline
			Conv\_128\_3x3 & 24x24x128\\\hline
			MaxPool\_2x2 & 12x12x128\\\hline
			
			Conv\_128\_4x4 & 9x9x128\\\hline
			Dense\_1\_output & 1\\
			\hline
		\end{tabular}
		\caption{Description of the architecture used in the experiments. For convolutional layers, Conv\_N\_WxH, where N is the number of filters, W the filter width and H the filter height. Stride is 1 for every convolutional layer. For max pool layers, MaxPool\_SxS, where S is the pool size.}
		\label{table:CNNArchitecture}
	\end{table}
	
	Every convolutional layer is followed by an ELU activation layer \cite{clevert2015fast} and a batch normalization \cite{ioffe2015batch}. At the output, a Proportional Odds Model is used with different link functions \cite{agresti2010analysis}. The logit link function is commonly used within POM. In this paper, we are comparing other link functions like probit or complementary log-log with the logit link. These three types of links are explained below.
	
	\begin{itemize}
		\item \textit{Logit}. Logit link function is the most widely used function for Proportional Odds Models. These kind of models are also called Cummulative Logit Models. The logit link is shown in Eq. \ref{eq:logit}.
		\begin{equation}
		\small
		logit[P(Y_i \le j)] = \alpha_j + \beta'x_i,\quad j = 1, ..., c - 1
		\label{eq:logit}
		\end{equation}
		\item \textit{Probit}. Probit link function is the inverse of the standard normal cummulative distribution function (cdf). Its expression is shown in Eq. \ref{eq:probit}.	
		\begin{equation}
		\small
		\Phi^{-1}[P(Y \le j)] = \alpha_j + \beta'x,\quad j = 1, ..., c - 1
		\label{eq:probit}
		\end{equation}
		\item \textit{Complementary log-log}. Unlike logit and probit, complementary log-log function is not symmetric. With a continuous predictor $x$, for example, $P(Y \le j)$ approaches 0 at a different rate than it approaches 1. Complementary log-log expression is shown in Eq. \ref{eq:cloglog}.
		\begin{equation}
		\footnotesize
		log[-log[1 - P(Y \le j)]] = \alpha_j + \beta'x,\quad j = 1, ..., c - 1
		\label{eq:cloglog}
		\end{equation}
	\end{itemize}
	
	\subsection{Procedure}
	The model is optimized using a batch based first-order optimization algorithm called Adam \cite{kingma2014adam}. We study different initial learning rates in order to find the optimal one for each problem. We apply an exponential decay across training epochs to the initial learning rate.
	
	Quadratic Weighted Kappa Loss is considered as loss function for this optimizer as it gives better performance for ordinal classification problems.
	
	Both datasets have been artificially equalised using data augmentation techniques \cite{van2001art}\cite{krizhevsky2012imagenet} based on image cropping and zooming, horizontal and vertical flipping, brightness adjustment and random rotations. In the case of Diabetic Retinopathy Detection, the epoch size has been fixed to $100000$ images per epoch. For the Adience dataset, epoch size is the number of images in the training set.
	
	The model is evaluated with Quadratic Weighted Kappa metric (QWK) \cite{ben2008comparison}. This evaluation measure gives a higher weight to the errors that are further from the correct class.
	
	 Experiments were run with the standard cross-entropy loss and the softmax function too in order to prove the performance improvement of the QWK loss and the Proportional Odds Model. The results of these experiments are analysed in Appendix \ref{app:NominalComparison}.
	
	\subsection{Parameters}
	Three different parameters have been considered: learning rate, batch size and link function for the final output layer.
	
	\begin{itemize}
		\item \textit{Learning rate}. Learning rate is one of the most critical hyper-parameters to tune for training deep neural networks. Optimal learning rate can vary depending on the dataset and the CNN architecture. Previous works have presented some techniques that adjust this parameter in order to achieve better performance \cite{smith2017cyclical}\cite{senior2013empirical}. Within this work, we have considered three different values for this parameter: $10^{-2}$, $10^{-3}$ and $10^{-4}$.
		\item \textit{Batch size}. Batch size is also an important parameter as it controls the number of weight updates that are made on every epoch. It can affect the training time and the model performance. In this paper, we have tried three separate batch sizes: $5$, $10$ and $15$.
		\item \textit{Link function}. Different link functions have been used for the POM at the last layer output: logit, probit and complementary log-log.
	\end{itemize}
	
	\section{Results}
	\label{sect:results}
	In this section, we present the results of the experiments. For each dataset, we show a table with the detailed experiments done training the model with each combination of parameters. Every parameter combination has been run five times. These tables show the average quadratic weighted kappa across these five executions for validation and test values.
	
	\subsection{Diabetic Retinopathy}
	Detailed results for the Diabetic Retinopathy dataset are presented in Table \ref{table:DRresults}.
	
	\begin{table}[ht]
		\footnotesize
		\centering
		\begin{tabular}{ccc|cc}
			BS & LR & LF & $\kappa_{val}$ & $\kappa_{test}$\\\hline\addlinespace[0.05cm]
			$5$ & $10^{-2}$ & Logit & $0.44888$ & $0.4163$\\
			& & Probit & $0.46724$ & $0.45972$\\
			& & c log-log & $0.42854$ & $0.41448$\\
			& $10^{-3}$ & Logit & $0.56496$ & $0.55356$\\
			& & Probit & $0.57084$ & $0.56392$\\
			& & c log-log & $0.54796$ & $0.53436$\\
			& $10^{-4}$ & Logit & $0.52884$ & $0.52032$\\
			& & Probit & $0.53802$ & $0.52302$\\
			& & c log-log & $0.53824$ & $0.51974$\\
			$10$ & $10^{-2}$ & Logit & $0.5497$ & $0.53142$\\
			& & Probit & $0.52124$ & $0.50806$\\
			& & c log-log & $0.4276$ & $0.4227$\\
			& $10^{-3}$ & Logit & $0.58036$ & $0.57686$\\
			& & Probit & $0.56536$ & $0.5581$\\
			& & c log-log & $\textbf{0.5883}$ & $\textbf{0.5815}$\\
			& $10^{-4}$ & Logit & $0.53158$ & $0.5385$\\
			& & Probit & $0.5456$ & $0.54082$\\
			& & c log-log & $0.53928$ & $0.53742$\\
			$15$ & $10^{-2}$ & Logit & $0.55828$ & $0.55076$\\
			& & Probit & $0.54112$ & $0.5343$\\
			& & c log-log & $0.56676$ & $0.56428$\\
			& $10^{-3}$ & Logit & $0.55748$ & $0.55106$\\
			& & Probit & $0.58182$ & $0.57894$\\
			& & c log-log & $0.5634$ & $0.55938$\\
			& $10^{-4}$ & Logit & $0.54686$ & $0.54342$\\
			& & Probit & $0.53258$ & $0.53278$\\
			& & c log-log & $0.539$ & $0.53828$
		\end{tabular}
		\caption{Diabetic Retinopathy results. BS stands for Batch Size, LR for Learning Rate and LF for link function.}
		\label{table:DRresults}
	\end{table}
	
	The best mean WK value was obtained with the complementary log-log link function using a batch size of 10 and a learning rate of $10^{-3}$.
	
	\subsection{Adience}
	Results for the experiments made with the Adience dataset are shown in Table \ref{table:Adienceresults}.
	
	\begin{table}[ht]
		\footnotesize
		\centering
		\begin{tabular}{ccc|cc}
			BS & LR & LF & $\kappa_{val}$ & $\kappa_{test}$\\\hline\addlinespace[0.05cm]
			$5$ & $10^{-2}$ & Logit & $0.72072$ & $0.68408$\\
			& & Probit & $0.77608$ & $0.72574$\\
			& & c log-log & $0.7543$ & $0.70742$\\
			& $10^{-3}$ & Logit & $0.7146$ & $0.67868$\\
			& & Probit & $0.81906$ & $0.77154$\\
			& & c log-log & $0.57196$ & $0.55608$\\
			& $10^{-4}$ & Logit & $0.54432$ & $0.51934$\\
			& & Probit & $0.5436$ & $0.52422$\\
			& & c log-log & $0.54802$ & $0.52484$\\
			$10$ & $10^{-2}$ & Logit & $0.78872$ & $0.72994$\\
			& & Probit & $0.79332$ & $0.7475$\\
			& & c log-log & $0.78004$ & $0.7326$\\
			& $10^{-3}$ & Logit & $0.49686$ & $0.47112$\\
			& & Probit & $0.59778$ & $0.58476$\\
			& & c log-log & $0.52016$ & $0.49968$\\
			& $10^{-4}$ & Logit & $0.60972$ & $0.56814$\\
			& & Probit & $0.54638$ & $0.52632$\\
			& & c log-log & $0.75428$ & $0.52094$\\
			$15$ & $10^{-2}$ & Logit & $0.79104$ & $0.74388$\\
			& & Probit & $\textbf{0.84254}$ & $\textbf{0.78666}$\\
			& & c log-log & $0.78802$ & $0.74624$\\
			& $10^{-3}$ & Logit & $0.56946$ & $0.52978$\\
			& & Probit & $0.70262$ & $0.65416$\\
			& & c log-log & $0.49242$ & $0.4732$\\
			& $10^{-4}$ & Logit & $0.55202$ & $0.52484$\\
			& & Probit & $0.54288$ & $0.5174$\\
			& & c log-log & $0.53826$ & $0.52542$
		\end{tabular}
		\caption{Adience results. BS stands for Batch Size, LR for Learning Rate and LF for link function.}
		\label{table:Adienceresults}
	\end{table}
	
	The best mean WK value was obtained with the probit link function using a batch size of 15 and a learning rate of $10^{-2}$.
	
	\subsection{Statistical analysis}
	In this subsection, a statistical analysis will be performed in order to obtain robust conclusions from the results presented in this section.
	
	The significance and relative importance of the parameters concerning the results obtained, as well as suitable values for each, were obtained using the ANalysis Of the VAriance (ANOVA) test.
	
	The ANOVA test \cite{miller1997beyond} is one of the most widely used statistical techniques. ANOVA is essentially a method of analysing the variance to which a response is subject into its various components, corresponding to the sources of variation which can be identified.
	
	ANOVA, in this case, examines the effects of three quantitative variables (termed factors) on one quantitative response. Considered factors are the link function, which can be logit, probit and complementary log-log, the learning rate for the Adam optimization algorithm ($10^{-2}$, $10^{-3}$ and $10^{-4}$) and the batch size (5, 10 and 15).
	
	Following the setup of the previous study, we performed an ANOVA III analysis and multiple comparison tests. We assume that five executions are enough to do the statistical tests because of the computational time limitations.
	
	The test shows that there is no batch size whose results are significatively better than the results of all other batch sizes. This does not mean that these differences could not exist for specific numbers of samples per batch. So, in order to determine for each type of function whether a batch size is better than the others, we have performed an ANOVA I analysis - where the only factor is the batch size - and multiple comparison tests.
	
	We denote by $W_{i,j,k,l}(i=1, ..., 3; j = 1, ..., 3; k = 1, ..., 3)$ the value observed when the first factor is at the i-th level, the second at the j-th level and the third at k-th level. We assume that the two factors do not act independently and therefore there exists an interaction between them. In this case, the observations fit Eq. \ref{eq:anova3}.
	
	\begin{equation}
	\begin{aligned}
	\footnotesize
	W_{i,j,k,l} = & \mu + L_i + P_j + B_k + LP_{i,j} + LB_{i,k} \\& + PB_{j,k} + LPB_{i,j,k} + \epsilon_{i,j,k,l}
	\label{eq:anova3}
	\end{aligned}
	\end{equation}
	
	where $\mu$ is the fixed effect that is common to all the populations; $L_i$ is the effect associated with the \textit{i-th} level of the link factor (logit, probit, complementary log-log); $P_j$ is the effect associated with the \textit{j-th} level of the parameter factor ($10^{-4}$, $10^{-3}$, $10^{-2}$) and $B_k$ is the effect associated with the \textit{k-th} level of the batch size factor (5, 10, 15). The term $LP_{i,j}$ denotes the joint effect of the presence of level $i$ of the first factor and level $j$ of the second one; this, therefore, is denominated the interaction term between $L$ and $P$ factors. The same interaction effect is appreciated on $LB_{i,k}$, $PB_{j,k}$ and $LPB_{i,j,k}$. The term $\epsilon_{i,j,k,l}$ is the influence on the result of everything that could not be assigned or of random factors. $W_{i,j,k,l}$ is the quadratic weighted kappa measure, the response variable used to perform the statistical analysis.
	
	We consider some hypotheses testing where the null hypothesis is proposed that each term of the above equation is independent of the levels involved. The hypotheses for the levels of the L factor are
	\[
		H_0 \equiv L_1 = L_2 = L_3
	\]
	\[
		H_1 \equiv \text{ some } L_i \text{ is different}
	\]
	
	The same hypotheses are made for the other factors. In this way, we test in the null hypothesis that all of the population means are equal against an alternative hypothesis that there is at least one mean that is not equal to the others.
	
	The hypothesis associated with the interaction between L and P is
	\[
		H_0 \equiv LP_{i,j} = 0 \quad \forall i,j
	\]
	\[
		H_1 \equiv \exists LP_{i,j} \ne 0
	\]
	
	Similar hypotheses can be assumed for the interaction between the other factors.
	
	The analysis of variance table containing the sum of squares, degrees of freedom, mean square, test statistics and significance level represents the initial study in a compact form.
	
	\begin{table}[ht]
		\centering
		\small
		\begin{tabular}{l|ccccc}
			\multicolumn{6}{c}{Response variable WK}\\\hline
			Source & S.S. & D.F. & M.S. & F-ratio & Sig.\\\hline
			Model & $37.860$ & $9$ & $4.207$ & $1562.840$ & $.000$\\
			$L$ factor & $.057$ & $2$ & $.029$ & $10.646$ & $.000$\\
			$P$ factor & $.121$ & $2$ & $.060$ & $22.468$ & $.000$\\
			$LP$ factors & $.057$ & $4$ & $.014$ & $5.261$ & $.001$\\
			Error & $.339$ & $126$ & $.003$ & & \\
			Total & $38.199$ & $135$ & & & 
		\end{tabular}
		\caption{ANOVA III for the analysis of the main factors in the design of a Convolutional Ordinal Neural Network for the Retinopathy dataset.}
		\label{table:ANOVARetinopathy}
	\end{table}
	
	The results of the ANOVA III test for the Diabetic Retinopathy dataset are resumed in Table \ref{table:ANOVARetinopathy}. There are significative differences in the means depending on the link function and also depending on the learning rate for $\alpha=0.05$. Moreover, an interaction between the link function and the learning rate can be recognised ($\text{p-value} = 0.001$).
	
	As Table \ref{table:ANOVARetinopathy} showed that there exist significative differences between the means, we are now analysing those differences. A post-hoc multiple comparison test has been performed on the average WK obtained. An HSD Tukey's test was made under the null hypothesis that the variance of the error of the dependent variable is the same between groups. The best link function is the complementary log-log, though it doesn't have significative differences with the probit function ($\alpha=0.05$). There are differences between the complementary log-log and the probit function concerning the logit function.
	
	The results have been obtained following a similar methodology with the different values of the learning parameter including the HSD Tukey's test and the learning parameter ranking. Significative differences can be observed in the mean values for $\eta = 10^{-3}$ with respect to the other values of the parameter ($\alpha=0.05$). Also, there are differences for $\eta = 10^{-4}$ concerning $\eta = 10^{-2}$.
	
	\begin{table}[ht]
		\centering
		\small
		\begin{tabular}{l|ccccc}
			\multicolumn{6}{c}{Response variable WK}\\\hline
			Source & S.S. & D.F. & M.S. & F-ratio & Sig.\\\hline
			Model & $52.328$ & $15$ & $3.489$ & $1426.170$ & $.000$\\
			$L$ factor & $.027$ & $2$ & $.014$ & $5.582$ & $.005$\\
			$P$ factor & $1.031$ & $2$ & $.516$ & $210.809$ & $.000$\\
			$B$ factor & $.089$ & $2$ & $.045$ & $18.253$ & $.000$\\
			$LP$ factors & $.183$ & $4$ & $.046$ & $18.695$ & $.000$\\
			$PB$ factors & $.124$ & $4$ & $.031$ & $12.695$ & $.000$\\
			Error & $.294$ & $120$ & $.002$ & & \\
			Total & $52.622$ & $135$ & $3.489$ & & 
		\end{tabular}
		\caption{ANOVA III for the analysis of the main factors in the design of a Convolutional Ordinal Neural Network for the Adience dataset.}
		\label{table:ANOVAAdience}
	\end{table}
	
	The results of the ANOVA III test for the Adience dataset are shown in Table \ref{table:ANOVAAdience}. The interactions where there are not significative differences have been omitted. First, the factor associated with the link function is analysed. The differences between the means are significative ($\text{p-value} = 0.005$). Secondly, there are significative differences concerning the learning rate factor too ($\text{p-value} = 0.000$). Also, there are differences for the means for the batch size factor. Significative interactions exist between two pairs of factors: link function and learning rate, and learning rate and batch size ($\text{p-value = 0.000}$ for both cases).
	
	A post-hoc multiple comparison test has been performed on the average WK obtained for this dataset too. Under the null hypothesis that the variance of the error of the dependent variable is the same between groups, an HSD Tukey's test has been done. The results showed that the best link function is the logit one, in this case, with $0.632438$ as mean WK value. However, there are not significative differences between this function and the complementary log-log ($\text{p-value}=0.110$). It has differences with the probit function though ($\text{p-value} = 0.003$). The complementary log-log link reported the best results after the logit function, having a mean WK of $0.611287$. However, it doesn't show significative differences with the probit function.
	The results have been obtained following a similar methodology with the different values of the learning parameter including the HSD Tukey's test and the learning parameter ranking. The test reported significative differences between $\eta=10^{-2}$ (mean value $0.733784$) and the rest of values of this parameter ($\alpha=0.005$), being this one the best value for this factor. The best value after $10^{-2}$ is $10^{-3}$, which have a mean WK value of $0.579889$. Lastly, the best value for the batch size parameter is $10$ (mean value $0.648700$) as it has significative differences with the rest of values. The second best value is $5$ (mean $0.605533$), but it has no significative differences with $15$ ($\text{p-value}= 0.194$).
	
	\section{Conclusions}
	\label{sect:conclusions}
	In this section, the conclusions obtained from this work are exposed. The first thing that we have noticed is that the optimal values for the different parameters considered are problem-dependant. 
	
	The complementary log-log function has the best average performance across both datasets. It offers the best results in Diabetic Retinopathy dataset and the second best result in Adience dataset. These results provide an opportunity for exploring new generalised link functions.
	
	The best value for the learning rate parameter for  Diabetic Retinopathy dataset is $\eta = 10^{-3}$ while this value is the second best option for Adience dataset. It can be considered a good value for this parameter when training the model with new datasets.
	
	Both datasets have obtained the best performance with batch size 10.
	
	Also, the statistical tests reported that there are interactions between some parameters like the link function and the learning rate, for Diabetic Retinopathy, or those factors in addition to learning rate and batch size, for Adience dataset.
	
	The proposed POM model has improved the performance of the deep network regarding Weighted Kappa compared to the model that uses the softmax function. This enhancement is more notable for the Diabetic Retinopathy dataset. Also, it reduces the chance that the model gets stuck when training with some parameter configurations. So, the most significant improvements of these link functions are the WK performance increase, the reduction of the number of parameters configurations that should be tried to find the best one and the prevention of the over-fitting and the stagnation.
	
	\printbibliography
	
	\appendix
	\section{Statistical comparison between nominal and ordinal methods}
	\label{app:NominalComparison}
	
	The same experiments described in section \ref{sect:experiments} were repeated using the cross-entropy instead of the QWK as loss function for the optimiser and the softmax function instead of the Proportional Odds Model for the output of the network.
	The results for both datasets are shown in Table \ref{table:NominalResults}.
	
	\begin{table}[ht]
		\footnotesize
		\centering
		\begin{tabular}{lcc|cc}
			Dataset & BS & LR & $\kappa_{val}$ & $\kappa_{test}$\\\hline\addlinespace[0.05cm]
			Retinopathy & $5$ & $10^{-2}$ & $0.2210$ & $0.2050$\\
			Retinopathy & $5$ & $10^{-3}$ & $0.2807$ & $0.2654$\\
			Retinopathy & $5$ & $10^{-4}$ & $0.4580$ & $0.4449$\\
			
			Retinopathy & $10$ & $10^{-2}$ & $0.2972$ & $0.2997$\\
			Retinopathy & $10$ & $10^{-3}$ & $0.3972$ & $0.3989$\\
			Retinopathy & $10$ & $10^{-4}$ & $\textbf{0.4972}$ & $\textbf{0.4967}$\\
			
			Retinopathy & $15$ & $10^{-2}$ & $0.3692$ & $0.3676$\\
			Retinopathy & $15$ & $10^{-3}$ & $0.4107$ & $0.4162$\\
			Retinopathy & $15$ & $10^{-4}$ & $0.4929$ & $0.4859$\\
			\hline\addlinespace[0.05cm]
			Adience & $5$ & $10^{-2}$ & $0.0605$ & $0.0769$\\
			Adience & $5$ & $10^{-3}$ & $0.7994$ & $0.7313$\\
			Adience & $5$ & $10^{-4}$ & $0.8009$ & $0.7383$\\
			
			Adience & $10$ & $10^{-2}$ & $0.0559$ & $0.0576$\\
			Adience & $10$ & $10^{-3}$ & $0.8239$ & $0.7436$\\
			Adience & $10$ & $10^{-4}$ & $0.7938$ & $0.7218$\\
			
			Adience & $15$ & $10^{-2}$ & $0.0593$ & $0.0713$\\
			Adience & $15$ & $10^{-3}$ & $\textbf{0.8309}$ & $\textbf{0.7575}$\\
			Adience & $15$ & $10^{-4}$ & $0.7867$ & $0.7129$\\
		\end{tabular}
		\caption{Adience results. BS stands for Batch Size, LR for Learning Rate and LF for link function.}
		\label{table:NominalResults}
	\end{table}
	
	As we did in Section X, we must analyse the effect of the batch size (with values 5, 10 and 15) and the learning rate ($10^{-2}$, $10^{-3}$ and $10^{-4}$) factors have over the WK metric. So, we make an ANOVA II analysis for each dataset.
	
	In the Diabetic Retinopathy dataset, there are significative differences between the mean values of WK depending on the batch size and the initial value of the learning rate. Also, no significative interaction was detected between these factors.
	
	After that, a post-hoc Tukey's test shows that the best value for the batch size factor is 15 followed by 10. However, there are no significative differences between them, but there are differences with size 5. Also, the best value for the learning rate is $10^{-4}$ and there are significative differences with the rest of values.
	
	A similar study was made for the Adience dataset. There are not significative differences for the mean value of WK w.r.t. the batch size factor. Also, there are no interactions between the batch size and the learning rate. The best results were obtained with $10^{-3}$ for the learning rate followed by $10^{-4}$, without significative differences between them.
	
	Finally, the conclusions of this study are exposed below:
	\begin{itemize}
		\item There are no interactions between both factors.
		\item Batch sizes 10 and 15 obtained the best results.
		\item Learning rate $10^{-4}$ is the best choice for both datasets as it the best for Diabetic Retinopathy, with significative differences, and it is the second for Adience, without significative differences with the best. 
	\end{itemize}
	
\end{document}