% !TeX spellcheck = en_UK
\documentclass[10pt, a4paper, titlepage]{article}

\usepackage[utf8]{inputenc}
\usepackage[english]{babel}
\usepackage{amsmath}
\usepackage{amsfonts}
\usepackage{amssymb}
\usepackage{graphicx}
\usepackage{url}
\usepackage{hyperref}
\usepackage{ragged2e}
\usepackage{tocbibind}
\usepackage{float}
\usepackage{placeins}
\usepackage{fancyhdr}
\usepackage{datetime}
\usepackage[dvipsnames,table]{xcolor}
\usepackage[backend=bibtex, style=numeric, sorting=none, maxbibnames=6, style=ieee]{biblatex}
\usepackage{geometry}
\usepackage[pages=some]{background}
\usepackage{setspace}
\usepackage{anysize}
\usepackage{smartdiagram}
\usepackage{tabulary}
\usepackage{booktabs}
\usepackage{multirow}
\usepackage{wrapfig}
\usepackage{csquotes}
\usepackage{longtable}
\usepackage{tocloft}
\usepackage{appendix}
\usepackage{multicol}

\author{}
\title{}
\date{Córdoba, \today}

\newdateformat{monthyeardate}{\monthname[\THEMONTH] de \THEYEAR}

%marginsize{left}{right}{top}{bottom}
\marginsize{2cm}{2cm}{2.5cm}{3.5cm}

\addbibresource{main.bib}

\setlength{\columnsep}{0.7cm}

\begin{document}
	\fontfamily{lmr}\selectfont
	
	\begin{multicols}{2}
	
	\section{Experiments}
	\subsection{Data}
	We make use of two ordinal datasets appropriate for deep neural networks:
	
	\begin{itemize}
		\item \textit{Diabetic Retinopathy}\footnote{https://www.kaggle.com/c/diabetic-retinopathy-detection/data}. This is a dataset consisting of extremely high-resolution fundus image data. The training set consists of $17563$ pairs of images (where a  pair  consists of a left and right eye image corresponding to a patient). In this dataset, we try and predict from five levels of diabetic retinopathy: no DR ($25810$ images), mild DR ($2443$ images), moderate DR ($5292$ images), severe DR ($873$ images), or proliferative DR ($708$ images). The images are taken in variable conditions: by different cameras, illumination conditions and resolutions. These images come from the EyePACS dataset that was used in a Diabetic Retinopathy Detection competition that was hosted on the Kaggle platform. Also, this dataset was used in later works \cite{de2018weighted} and ordinal techniques (such as an ordinal cost function) were applied in order to achieve a better performance. A validation set is set aside, consisting of $10\%$ of the patients in the training set. The images are resized to 256 by 256 pixels. Data augmentation techniques are applied in order to achieve a higher number of samples.
		
		\item \textit{Adience}\footnote{http://www.openu.ac.il/home/hassner/Adience/data.html}. This dataset consists of $26580$ faces belonging to $2284$ subjects. We use the form of the dataset where faces have been pre-cropped and aligned. The dataset was preprocessed, using the methods described in a previous work \cite{beckham2017unimodal}, so that the images are 256px in width and height and pixels values follow a $(0;1)$ normal distribution. The original dataset is splitted in 5 cross-validation folds. The training set consists of merging the first four folds together which comprises a total of $15554$ images. From this, $10\%$ of the images are held out as part of a validation set. The last fold is used as test set.
	\end{itemize}
	
	\subsection{The model}
	A convolutional neural network (CNN) has been used for both datasets. The architecture of this CNN is presented in the Table \ref{table:CNNArquitecture}.
	
	\begin{table}[H]
		\centering
		\begin{tabular}{|l|c|}
			\hline
			Layer & Output shape\\
			\hline
			Conv\_32\_3x3 & 254x254x32\\\hline
			Conv\_32\_3x3 & 252x252x32\\\hline
			MaxPool\_2x2 & 126x126x32\\\hline
			
			Conv\_64\_3x3 & 124x124x64\\\hline
			Conv\_64\_3x3 & 122x122x64\\\hline
			MaxPool\_2x2 & 61x61x64\\\hline
			
			Conv\_128\_3x3 & 59x59x128\\\hline
			Conv\_128\_3x3 & 57x57x128\\\hline
			MaxPool\_2x2 & 28x28x128\\\hline
			
			Conv\_128\_3x3 & 26x26x128\\\hline
			Conv\_128\_3x3 & 24x24x128\\\hline
			MaxPool\_2x2 & 12x12x128\\\hline
			
			Conv\_128\_4x4 & 9x9x128\\\hline
			Dense\_1\_output & 1\\
			\hline
		\end{tabular}
		\caption{Description of the architecture used in the experiments. For convolutional layers, Conv\_N\_WxH, where N is the number of filters, W the filter width and H the filter height. Stride is 1 for every convolutional layer. For max pool layers, MaxPool\_SxS, where S is the pool size.}
		\label{table:CNNArquitecture}
	\end{table}
	
	Every convolutional layer is followed by an ELU activation layer \cite{clevert2015fast} and a batch normalization \cite{ioffe2015batch}. At the output, a Proportional Odds Model is used with different link functions (logit, probit, complementary log-log).
	
	\printbibliography
	
	\end{multicols}
\end{document}